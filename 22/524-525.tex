\documentclass{article}
\usepackage[utf8]{inputenc}
\usepackage[russian]{babel}
\usepackage{setspace,amsmath}
\usepackage[14pt]{extsizes}
\usepackage[left=3cm,right=3cm,
    top=3cm,bottom=3cm,bindingoffset=0cm]{geometry}
\setcounter{page}{524}
\begin{document}
 
 
отсюда $\displaystyle {x} = {(1 - z^4)}^{-2/3},{dx}={{8}\over{3}}{(1 - z^4)}^{-5/3}{z^3dz}$, поэтому
    \begin{gather*}
    I={{8}\over{3}}\int_{}^{}{{z^4}\over{(1-z^4)^2}}dz={{2}\over{3}}\int_{}^{}z \ d{{1}\over{1-z^4}}={2\over3}({{z}\over{1-z^4}}-\int_{}^{}{{dz}\over{1-z^4}})=
    \\
    ={{2z}\over{3(1-z^4)}}-{{1}\over{3}}\int_{}^{}({{1}\over{1-z^2}}+{{1}\over{1+z^2}})dz=
    \\
    ={{2z}\over{3(1-z^4)}}-{1\over6}ln \ |{{1+z}\over{1-z}}|-{1\over3}arctg \ z + C,
    \end{gather*}
    где $\displaystyle z$ выражается через $\displaystyle x$ по формуле (21.15).
    \\


\textbf{21.15 Интегралы вида $\displaystyle \int_{}^{} {{P_n(x)dx}\over{\sqrt{ax^2+bx+c}}}$} \\ \\ Рассмотрим интеграл
\begin{gather*} \int_{}^{} {{P_n(x)dx}\over{\sqrt{ax^2+bx+c}}}dx, \ a \ne 0, \end{gather*}
где $\displaystyle P_n(x)dx$ - многочлен степени $\displaystyle n \geq 1$. С принципиальной точки зрения этот интеграл всегда можно свести к интегралу от рациональной дроби с помощью одной из подстановок Эйлера (см.п.21.3). Однако в данном конкретном случае значительно быстрее к цели приводит обычно другой прием. \par
Именно, покажем, что справедлива формула
\begin{gather*} \int_{}^{} {{P_n(x)dx}\over{\sqrt{ax^2+bx+c}}}dx=
\\
={P_{n-1}(x)}\sqrt{ax^2+bx+c}+\alpha \int_{}^{} {{dx}\over{\sqrt{ax^2+bx+c}}}, \ \ (21.16) \end{gather*}
где $\displaystyle P_{n-1}(x)dx$ - многочлен степени не выше, чем $\displaystyle n-1$, а $\displaystyle \alpha $ - некоторое число.
\par
Итак, пусть многочлен
\begin{gather*} P_{n}(x)=a_nx^n+a_{n-1}x^{n-1}+\ldots +a_0 \ \ (21.17)\end{gather*}
задан. Если существует многочлен
\begin{gather*} P_{n-1}(x)=b_{n-1}x^{n-1}+b_{n-2}x^{n-2}+\ldots+b_0,\ \ (21.18)\end{gather*}
удовлетворяющий условию (21.16), то, дифференцируя это равенство, получим
\begin{gather*} {{P_n(x)}\over{\sqrt{ax^2+bx+c}}}=
\\
=P_{n-1}^{'}(x)\sqrt{ax^2+bx+c}+{{P_{n-1}(x)(2ax+b)}\over{2\sqrt{ax^2+bx+c}}}+{{\alpha}\over{\sqrt{ax^2+bx+c}}},\end{gather*}
или
\begin{gather*} 2P_{n}(x)=2P_{n-1}^{'}(x)(\sqrt{ax^2+bx+c})+P_{n-1}(x)(2ax+b)+2\alpha. \ \ (21.19)\end{gather*}


 
Здесь слева стоит многочлен степени $\displaystyle n$, а справа каждое слагаемое также является многочленом степени не больше $\displaystyle n$.
\par
Замечая, что
\begin{gather*} P_{n-1}^{'}=
=(n-1)b_{n-1}x^{n-2}+\ldots +kb_{k}x^{k-1}+\ldots+b_1, \ \ (21.20)\end{gather*}
и подставляя (21.17), (21.18) и (21.20) в (21.19), имеем равенства
\begin{gather*} 2(a_nx^n+a_{n-1}x^{n-1}+\ldots +a_1x+a_0)=
\\
=2(ax^2+bx+c)[(n-1)b_{n-1}x^{n-2}+\ldots +kb_{k}x^{k-1}+\ldots+b_1]+
\\
+(2ax+b)(b_{n-1}x^{n-1}+\ldots +b_{k}x^{k}+\ldots +b_0)+2\alpha\end{gather*}
\ \ \ \ Приравнивая коэффиценты у одинаковых степеней $\displaystyle x$, получим следующую систему $\displaystyle n+1$ линейных уравнений с $\displaystyle n+1$ неизвестными $\displaystyle b_0,b_1,\ldots,b_{n-1},\alpha:$
\begin{gather*} 2a_0=2cb_1+bb_0+2\alpha,
\\
2a_1=2bb_1+4cb_2+2ab_0+bb_1,
\\
\ldots\ldots\ldots\ldots\ldots\ldots\ldots\ldots\ldots\ldots\ldots\ldots\ldots\ldots\ldots\ldots\ldots\ldots\ldots
\\
 2a_k=2(k-1)ab_{k-1}+2kbb_k+2(k+1)cb_{k+1}+2ab_{k-1}+bb_k,
\\
\ldots\ldots\ldots\ldots\ldots\ldots\ldots\ldots\ldots\ldots\ldots\ldots\ldots\ldots\ldots\ldots\ldots\ldots\ldots\ldots\ldots\ldots\ldots
\\
2a_{n-1}=2(n-2)ab_{n-2}+2(n-1)bb_{n-1}+2ab_{n-2}+bb_{n-1},
\\
2a_n=2(n-1)ab_{n-1}+2ab_{n-1}\end{gather*}

Из последнего уравнения сразу находим:$\displaystyle b_{n-1}=a_n/na$/. Подставляя это выражение в предпоследнее уравнение и замечая, что в этом уравнении коэффицент у неизвестного $\displaystyle b_{n-2}$ равен $\displaystyle 2a(n-1)\ne0$, найдем значение $\displaystyle b_{n-2}$. Подставляя далее значения $\displaystyle b_{n-1}$ и $\displaystyle b_{n-2}$ в предыдущее уравнение, найдем
 
  
  
\end{document}